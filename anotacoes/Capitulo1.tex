% !TeX spellcheck = pt_BR

\documentclass[Tese.tex]{subfiles}

\begin{document}

\chapter{Introdução}

\rightline
{\textit{``The mind that opens to a new idea never returns to its original size''}}
\rightline{Albert Einstein (1879 - 1955)}


IMAGENS

texto antes

%inclusão de gráficos (png, jpeg, pdf, ...)
% \usepackage{graphicx}

\begin{figure}[H]
	\includegraphics[width=10cm]{img/ilha.png}
\end{figure}	

% NOTA: [h] é uma recomendação ao motor TeX
% [H] força mesmo a posição.

texto depois

% https://pt.overleaf.com/learn/latex/How_to_Write_a_Thesis_in_LaTeX_(Part_3):_Figures,_Subfigures_and_Tables


% \label e \ref
% Como podemos ver na Fig.~\ref{fig:three sin x} 
% Como podemos ver na Fig.~1
% \label{fig:three sin x}  declara onde está a referência
% \ref{fig:three sin x}    utiliza a referência


\section{Antecedentes e motivação}

A educação é um processo dinâmico 


% \theoremstyle{plain}
% \newtheorem{theorem}{Teorema}[section]

\begin{theorem}\label{teo: importante}
	um pequeno teorema antes de um grande teorema [Altman, D.G. (1991)]
\end{theorem}


\begin{theorem}\label{teo: importante}
	um pequeno teorema antes de um grande teorema \cite{altman91}
\end{theorem}

% no ficheiro bilbliografia.tex deve estar
%\noindent
%\bibitem{altman91} Altman, D.G. (1991) \textit{Practical Statistics for %Medical Research}. New York: Chapman and Hall.

\begin{equation}\label{eq: principal}
	\frac{x+y}{2x}
\end{equation}

% para produzir aspas corretamente : ` ` texto  '  '

E depois no ``texto'' comparando com "texto"\  posso refe\-renciar a equação (\ref{eq: principal}).

% a palavra pode ser partida em \- :   refe\-renciar
% ajuda a produzir linhas partidas assim:
%                                                         refe  
% renciar



\begin{lemma}
	um pequeno teorema antes de um grande teorema~\ref{teo: importante} que está na página \pageref{teo: importante}.

\end{lemma}

% undefined control sequence
% \refpage não existe
% porque o que existe é
% \pageref


\begin{itemize}

\item  $-30\,^{\circ}\mathrm{C}$, Daqui a pouco estou a congelar. 

\item  $-30$~\textcelsius

\end{itemize}






\end{document}
