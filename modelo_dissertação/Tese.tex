
% Prof. Tomás Oliveira no DETI
% TeX  <= baixo nível      LaTeX <= utilizadores
\documentclass[a4paper,twoside,12pt,final]{report}

\def\ThesisYear{2021 (ver topo do ficheiro Tese.tex para mudar esta data)}



%%%%%%%%%%%%%%%%%%%%%%%%%%%%%%%%%%%%%%%%%%%%%%%%%%%%%%%%%%%%%%%%%%%%%%%%%%%

%O documento Tese.tex é o "master" (o principal).
%mas com este "subfiles", os capítulos podem ser compilados em separado
\usepackage{subfiles}


%%%%%%%%%%%%%%%%%%%%%%%%%%%%%%%%%%%%%%%%%%%%%%%%%%%%%%%%%%%%%%%%%%%%%%%%%%%

%Uma ou várias páginas PDF dentro da dissertação.
%\usepackage[final]{pdfpages}


%%%%%%%%%%%%%%%%%%%%%%%%%%%%%%%%%%%%%%%%%%%%%%%%%%%%%%%%%%%%%%%%%%%%%%%%%%%


%Colocar o texto dentro de colunas
\usepackage{multicol}


\usepackage{textcomp}
\usepackage{ragged2e}


%%%%%%%%%%%%%%%%%%%%%%%%%%%%%%%%%%%%%%%%%%%%%%%%%%%%%%%%%%%%%%%%%%%%%%%%%%%

%Uso de áéíóú etc com teclado português
% ASCII letras que usavam 7 bits => 128 símbolos diferentes (apenas)
% MSDOS/Windows => latin1 => já permite acentuação em várias línguas  (não é uniforme)
% unicode => "utf8" => standard da internet, linux, e que nem sempre está dispononível em app windows
% ERRO principal é que "ção" => "?????"
% regra: manter o mesmo computador e mesma impressora até ao final da dissertação!

%permite ler acentos do teclado português
\usepackage[utf8]{inputenc}  
% a alternativa:
% inovação => inova\c c\~ao  =>  \~a = ã   "\c c" => ç
% texto que mistura inglês com português.

%conselho:
% 1. tese.tex => tese_final.tex => tese_final_final2.tex => tese_final_final_da_final.tex
% 2. tese.tex e fazer cópias regulares tese-2021-03-30.tex, tese-2021-04-15.tex, etc
%     mantendo sempre o trabalho fresco em "tese.tex"
% 3. Guardar ficheiros no proprio email 
% 4. STIC: guardar 3 cópias em lugares distintos

%Gera letras com acentuação 
\usepackage[T1]{fontenc}

%Permite usar o \euro\ para desenhar o símbolo do euro.
% \euro\   => "\euro"  (desenha o símbolo; depois o "\ " assegura um espaço depois do símbolo euro.
\usepackage{eurosym}

%%%%%%%%%%%%%%%%%%%%%%%%%%%%%%%%%%%%%%%%%%%%%%%%%%%%%%%%%%%%%%%%%%%%%%%%%%%


%Quando se pretende alterar o formato de numeração dos "enumerates"
\usepackage{enumerate}


%%%%%%%%%%%%%%%%%%%%%%%%%%%%%%%%%%%%%%%%%%%%%%%%%%%%%%%%%%%%%%%%%%%%%%%%%%%


%package feito na Universidade de Aveiro com o formato exigido.
\usepackage{tese}


%%%%%%%%%%%%%%%%%%%%%%%%%%%%%%%%%%%%%%%%%%%%%%%%%%%%%%%%%%%%%%%%%%%%%%%%%%%

%
% AMS = American Mathematical Society 
%
\usepackage{amsfonts,amsmath,amssymb}
\usepackage{amstext,amscd,mathrsfs}


\usepackage{array}
\usepackage{empheq}


%%%%%%%%%%%%%%%%%%%%%%%%%%%%%%%%%%%%%%%%%%%%%%%%%%%%%%%%%%%%%%%%%%%%%%%%%%%

%inclusão de gráficos (png, jpeg, pdf, ...)
\usepackage{graphicx}

%permite \begin{figure}[H] <== o "H" indica que a figura não saia do local onde foi criada.
\usepackage{float}

%figuras com várias subfiguras (a), (b), (c), ....
\usepackage{subcaption}


%%%%%%%%%%%%%%%%%%%%%%%%%%%%%%%%%%%%%%%%%%%%%%%%%%%%%%%%%%%%%%%%%%%%%%%%%%%


\setcounter{MaxMatrixCols}{10}

%\usepackage[hmargin=2cm,vmargin=2cm,bmargin=1cm]{geometry}
\usepackage[  layoutwidth=210mm, 
              layoutheight=295mm,
              nomarginpar,
              nofoot,
              textwidth=170mm, 
              textheight=240mm, 
              inner=25mm, 
              verbose, 
              outer=20mm,  
              top=25mm,
              centering,
              footskip=22pt]{geometry}
              
\renewcommand{\baselinestretch}{1.5}

\theoremstyle{plain}
\newtheorem{theorem}{Teorema}[section]
\newtheorem{corollary}[theorem]{Corolário}
\newtheorem{lemma}[theorem]{Lema}
\newtheorem{proposition}[theorem]{Proposição}
\newtheorem{axiom}[theorem]{Axioma}
\theoremstyle{definition}
\newtheorem{definition}[theorem]{Definição}
\newtheorem{notation}[theorem]{Notação}
\newtheorem{example}[theorem]{Exemplo}
\newtheorem{remark}[theorem]{Nota}
\renewcommand{\thenotation}{}
%\renewcommand{\theequation}{\thechapter.\arabic{equation}}
%\numberwithin{equation}{section}

%Tabelas
\usepackage{longtable}
\renewcommand{\thetable}{\thesection.\arabic{table}}
\numberwithin{table}{section}

%Figuras
\renewcommand{\thefigure}{\thesection.\arabic{figure}}
%\numberwithin{figure}{section}
\numberwithin{figure}{chapter}
\newcommand{\mvec}[1]{\mbox{\bfseries\itshape #1}}


\usepackage{lscape}

\begin{document}%

\graphicspath{{./img/}}

%%%%%%%%%%%%%%%%%%%%%%%%%%%%%%%%%%%%%%%%%%%%%%%%%%%%%%%

\pagestyle{empty}

\TitlePage%
%\GRID  % for debugging ONLY
%   \HEADER{\BAR\FIG{\includegraphics[height=60mm]{ua30}}} % the \FIG{} is optional
%          {\ThesisYear}
%   \TITLE{Z\'{e} \newline Manel}
%         {Como escrever uma tese bonita e cheia de resultados importantes}
% \EndTitlePage
% %
% \titlepage%
% %\makeatletter%
% %      \if@twocolumn
% %        \@restonecoltrue\onecolumn
% %      \else
% %        \@restonecolfalse\newpage
% %      \fi
% %      \thispagestyle{empty}%
% %      \setcounter{page}\z@
% %\makeatother%
% \ %
% %
% \endtitlepage % empty page
% \newpage
% \TitlePage
  %\GRID  % for debugging ONLY
  \HEADER{\BAR%
%\FIG{%
%           \begin{minipage}{50mm} % no more than 120mm
%          ``I'm King of the world.''
%           \begin{flushright}
%            --- Jack Nicholson
%           \end{flushright}
%          \end{minipage}}
}
         {\ThesisYear}
  \TITLE{(Nome Do Autor)}
        {Uma Meta-Avalia\c c\~ao (título da disserta\c c\~ao)}
\EndTitlePage
%\titlepage\ \endtitlepage % empty page


%
% First thesis pages
%

\TitlePage
  \HEADER{}{\ThesisYear}
  \TITLE{(nome completo do aluno)}
        {Um Título}
  \vspace*{15mm}
  \TEXT{}{Disserta\c{c}\~{a}o apresentada \`{a} Universidade de Aveiro para cumprimento dos requisitos necess\'{a}rios
       \`{a} obten\c{c}\~{a}o do grau de Mestre em  Departamento de Ci\^encias M\'edicas, realizada sob a orienta\c{c}\~{a}o cient\'{\i}fica de (Nome do Professor 1) e co-orientação científica de (Nome do Professor 2), Professores Auxiliares do Departamento de Ci\^encias M\'edicas da Universidade de Aveiro.}
  %\vspace{50mm}
%  \TEXT{}
\EndTitlePage
%\titlepage\ \endtitlepage % empty page



\TitlePage
%\centering
  \vspace*{15cm}
  %\begin{flushright}
 \TEXT{Aos .... (\textit{in memoriam})}
     {}
    \vspace*{5mm}
 \TEXT{À ....}
     {}
    \vspace*{5mm}
 \TEXT{Aos ....}
     {}
    \vspace*{5mm}
 \TEXT{Algo mais?}
     {}
  %\end{flushright}
\EndTitlePage




\TitlePage
  \vspace*{55mm}
  \TEXT{\textbf{o j\'{u}ri~/~the jury\newline}}
       {}
  \TEXT{presidente~/~president}
       {\textbf{Doutor (nome do professor)}\newline 
         {\small Professor Auxiliar do Departamento de Ci\^encias M\'edicas da Universidade de Aveiro}
       }
  \vspace*{5mm}
  \TEXT{}
       {\textbf{Doutor (nome do professor)}\newline 
         {\small
        Professor Auxiliar Convidado ......Onde?}
       }
   \vspace*{5mm}
   \TEXT{}
        {\textbf{Doutor (nome do professor)}\newline 
           {\small
           Professora Auxiliar do Departamento de Ci\^encias M\'edicas da Universidade de Aveiro (orientador)}
        }
\EndTitlePage

       



%\titlepage\ \endtitlepage % empty page

\TitlePage
  \vspace*{55mm}
  \TEXT{\textbf{agradecimentos}}
       {À .............}
\vspace*{10mm}
\EndTitlePage



\TitlePage
  \vspace*{45mm}
  \TEXT{\textbf{Palavras-chave}}
  {Meta-avaliação,}
  \vspace{10mm}
  \TEXT{\textbf{Resumo}}
       {Seven hells. Ever vigilant, moon-flower juice green dreams, old bear arbor gold, though all men do despise us. Maecenas you know nothing.Seven hells. Ever vigilant, moon-flower juice green dreams, old bear arbor gold, though all men do despise us. Maecenas you know nothing.Seven hells. Ever vigilant, moon-flower juice green dreams,
       	old bear arbor gold, though all men do despise us. Maecenas you know nothing.Seven hells. Ever vigilant, moon-flower juice green dreams, old bear arbor gold, though all men do despise us. Maecenas you know nothing.Seven hells. Ever vigilant, moon-flower juice green dreams, old bear arbor gold, though all men do despise us. Maecenas you know nothing.}
  \TEXT{}
       {}
\vspace*{10mm}
\EndTitlePage


\TitlePage
  \vspace*{40mm}
  \TEXT{}
       {Espera-se que este estudo demonstre......}
  \TEXT{}
       {}
\vspace*{10mm}
\EndTitlePage



%\titlepage\ \endtitlepage % empty page


\TitlePage
  \vspace*{45mm}
  \TEXT{\textbf{Keywords}}
  {Meta-evaluation,}
  \vspace{10mm}
  \TEXT{\textbf{Abstract}}
       {Tests consist .\\[1ex]
        Three IRT models were ....}
  \TEXT{}
       {}
\vspace*{10mm}
\EndTitlePage


\TitlePage
  \vspace*{40mm}
  \TEXT{}
       {It is hoped that ......
}
  \TEXT{}
       {}
\vspace*{10mm}
\EndTitlePage



\pagenumbering{roman}
%\count0=0\relax\thispagestyle{empty}\mbox{}\newpage

\tableofcontents

\cleardoublepage

\listoffigures

\cleardoublepage

\listoftables

\cleardoublepage


\chapter*{Abreviaturas}

\begin{description}
\item[CCI]  {Curva ....}
\item[CII]  {Curva ....}
\item[CIT]  {Curva ....}
\item[DCM]  {Departamento de Ci\^encias M\'edicas}
\end{description}


\cleardoublepage





%%%%%%%%%%%%%%%%%%%%%%%%%%%%%%%%%%%%%%%%%%%%%%%%%%%%%%%


\clearpage{\thispagestyle{empty}}
\pagestyle{tese}
\pagenumbering{arabic}

%%%%%%%%%%%%%%%%%%%%%%%%%%%%%%%%%%%%%%%%%%%%%%%%%%%%%%%

\cleardoublepage
\subfile{Capitulo1}   %Introdução

%%%%%%%%%%%%%%%%%%%%%%%%%%%%%%%%%%%%%%%%%%%%%%%%%%%%%%%

\cleardoublepage  
\subfile{Capitulo2}

%%%%%%%%%%%%%%%%%%%%%%%%%%%%%%%%%%%%%%%%%%%%%%%%%%%%%%%

\cleardoublepage
\subfile{Capitulo3}

%%%%%%%%%%%%%%%%%%%%%%%%%%%%%%%%%%%%%%%%%%%%%%%%%%%%%%%

\cleardoublepage
\subfile{Capitulo4}

%%%%%%%%%%%%%%%%%%%%%%%%%%%%%%%%%%%%%%%%%%%%%%%%%%%%%%%

\cleardoublepage
\subfile{Capitulo5} %Conclusao

%%%%%%%%%%%%%%%%%%%%%%%%%%%%%%%%%%%%%%%%%%%%%%%%%%%%%%%%

\cleardoublepage



%\bibliographystyle{abbrv}
%\bibliographystyle{abbrvnat}
%\addcontentsline{toc}{chapter}{Bibliografia}
%\bibliography{bibliografia}
%\begin{thebibliography}{14}
%\addcontentsline{toc}{chapter}{Bibliografia}
%\providecommand{\natexlab}[1]{#1}
%\providecommand{\url}[1]{\texttt{#1}}
%\expandafter\ifx\csname urlstyle\endcsname\relax
%  \providecommand{\doi}[1]{doi: #1}\else
%  \providecommand{\doi}{doi: \begingroup \urlstyle{rm}\Url}\fi
%\bibitem[Biehler, 1989]{biehler}
%Biehler, ~R. (1989),
%\newblock Educational perspectives on exploratory data analysis, In R. Morris (Ed.)
%\newblock \emph{Studies in mathematics education: The teaching of statistics}, 7, pp. 185-202.
%\end{thebibliography}



\chapter*{Bibliografia}
\markboth{Bibliografia}{Bibliografia}


\noindent
Altman, D.G. (1991) \textit{Practical Statistics for Medical Research}. New York: Chapman and Hall.

\noindent
Andrade, D.F., Tavares, H.R., Valle, R.C. (2000) \textit{Teoria da Resposta ao Item: Conceitos e Aplicações}. São Paulo: Associação Brasileira de Estatística - ABE.

\noindent
Andrich, D.A. (1978) Rating Formulation for Ordered Response Categories. \textit{Psychometrika}, 43(4), 561-73.

\noindent
Baker, F.B. (1992) \textit{Item Response Theory - Parameter Estimation Techniques}. New York: Marcel Dekker, Inc.

\noindent
Bock, R.D., Lieberman, M. (1970) Fitting a Response Model for n Dichotomously Scored Items. \textit{Psychometrika}, 35(2), 179-197.

\noindent
Bock, R.D., Aitkin, M. (1981) Marginal Maximum Likelihood Estimation of Item Parameters: Application of an EM Algorithm. \textit{Psychometrika}, 46(4), 443-459.

\noindent
Burnham, K.P., Anderson, D.R. (2004) Multimodel Inference: Understanding AIC and BIC in Model Selection. \textit{Sociological Methods and Research}, 33(2), 261-304.

\noindent
Campbell, D.T., Stanley, J. (1973) \textit{Experimental and Quasi-Experimental Designs for Research}. Skokie, IL: Rand McNally.







%%%%%%%%%%%%%%%%%%%%%%%%%%%%%%%%%%%%%%%%%%%%%%%%%%%%%%%%

\cleardoublepage
\subfile{Apendices}
% Apêndice é o texto ou documento usado para complementar um trabalho, que foi elaborado pelo próprio autor.

%%%%%%%%%%%%%%%%%%%%%%%%%%%%%%%%%%%%%%%%%%%%%%%%%%%%%%%%

\cleardoublepage
\subfile{Anexos}
%Anexo é o texto ou documento usado para complementar um trabalho, que não foi elaborado pelo próprio autor dele.

%%%%%%%%%%%%%%%%%%%%%%%%%%%%%%%%%%%%%%%%%%%%%%%%%%%%%%%%

\end{document}

